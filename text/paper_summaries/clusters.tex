\documentclass{article}
    \usepackage{amssymb}
    \usepackage[utf8]{inputenc}
    \usepackage[russian]{babel}
    \usepackage[left=2cm,right=2cm,
        top=2cm,bottom=2cm,bindingoffset=0cm]{geometry}
    \usepackage{hyperref}
    \hypersetup{
        colorlinks=true,
        linkcolor=blue,
        filecolor=magenta,      
        urlcolor=cyan,
    }
  \usepackage{graphicx}
  \graphicspath{{pictures/}}
  \DeclareGraphicsExtensions{.pdf,.png,.jpg}

\begin{document}
\begin{center}{\huge Введение в предметную область: скопления галактик\\}\end{center}

\section{Clusters and superclusters in the Sloan Digital Sky Survey}
\hyperlink{https://www.aanda.org/articles/aa/pdf/2003/26/aah4162.pdf}{Статья}\\

Исследование данных SDSS с целью поиска скоплений и сверхскоплений для изучения структуры Вселенной 
c помощью метода поля плотности (density field method). Скопления определяются как увеличения поля 
плотности. Для отделения скоплений с различными параметрами размера и светимости используются 
различные параметры сглаживания. Для нахождения скоплений и групп галактик используется поле 
плотности с высоким разрешением, а для сверхскоплений - поле плотности с низким разрешением.\\ 
\end{document}
