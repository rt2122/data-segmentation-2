\Introduction

Галактики не расположены случайным образом в пространстве. Они формируют собой особые структуры, 
такие как скопления и сверхскопления галактик. Эти структуры в свою очередь формируют собой цепи, 
или так называемые "нити".\\

\begin{figure}
    \center{\includegraphics[width=0.6\linewidth]{model}}
    \caption{Моделирование «Милленниум» — N-частичное моделирование, проведённое Консорциумом 
        Девы с целью изучения формирования крупномасштабной структуры Вселенной в стандартной
        космологической модели.}
\end{figure}

Скопления галактик представляют большой интерес для исследования, так как их свойства сильно зависят 
от космологических параметров. Изучая их свойства, можно делать выводы о структуре обозримой части 
Вселенной.\\

Сама по себе крупномасштабная структура Вселенной имеет объяснение. При появлении Вселенной 
возмущения волн плотности средних и больших масштабов при совпадении пиков образовали сверхскопления,
в то время как сопадения фаз низкой плотности образовали войды - огромные пространства между нитями 
скоплений, в которых почти отсутствуют галактики и скопления. Таким образом, зная расположение и 
параметры большого количества скоплений, можно сделать выводы о том, как развивалась Вселенная на 
ранних этапах.\\

Одним из первых каталогов скоплений стал каталог Abell \cite{Abell}. Этот каталог содержит $4073$ 
богатых скопления галактик с красными смещениями $z < 0.2$. Он был построен при исользовании 
оптических данных обзора NGS-POSS.\\

\begin{figure}
    \center{\includegraphics[width=0.6\linewidth]{coma_sdss}}
    \caption{Скопление Волос Вероники (Abell 1656) в обзоре SDSS - одно из самых известных 
        скоплений каталога Abell}
\end{figure}

Позднее были созданы каталоги с использованием других диапазонов. Далее для исследования будут 
использоваться следующие каталоги:\\

\section{PSZ2}

\section{MCXC}

\section{RedMaPPer}

\section{ACT}

