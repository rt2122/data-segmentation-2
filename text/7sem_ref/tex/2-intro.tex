\Introduction

Галактики не расположены случайным образом в пространстве. Они формируют собой особые структуры, 
такие как скопления и сверхскопления галактик. Эти структуры в свою очередь формируют собой цепи, 
или так называемые "нити".\\

\begin{figure}
    \center{\includegraphics[width=0.6\linewidth]{millennium}}
    \caption{Моделирование «Милленниум» — N-частичное моделирование, проведённое Консорциумом 
        Девы с целью изучения формирования крупномасштабной структуры Вселенной в стандартной
        космологической модели.}
\end{figure}

Скопления галактик представляют большой интерес для исследования, так как их свойства сильно зависят 
от космологических параметров. Изучая их свойства, можно делать выводы о структуре обозримой части 
Вселенной.\\
