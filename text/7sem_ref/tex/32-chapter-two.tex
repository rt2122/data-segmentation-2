\chapter{Обзор данных}
\label{cha:ch_2}

Для создания тренировочной выборки использовались два каталога: часть каталога PSZ2 и каталог ACT.\\

\section{PSZ2}
\cite{Planck} Это каталог всего неба источников Сюняева-Зельдовича (SZ), обнаруженных 
по полным 29-месячным данным миссии Planck. Каталог (PSZ2) - это самая большая выборка скоплений 
галактик, отобранная по SZ, и самый глубокий систематический обзор скоплений галактик по всему 
небу. Он содержит 1653 обнаружения, из которых 1203 являются подтвержденными скоплениями с
идентифицированными аналогами во внешних наборах данных. В справочной статье авторы описывают 
многоволновой поиск аналогов во вспомогательных данных, который использует наборы радио-, 
микроволновых, инфракрасных, оптических и рентгеновских данных и делают упор на надежность 
сопоставления двойников. Они обсуждают физические свойства нового каталога и идентифицируют 
совокупность тусклых рентгеновских скоплений с малым красным смещением, выявленных с помощью 
SZ-отбора. Эти объекты появляются в оптических обзорах и обзорах SZ с одинаковыми характеристиками 
для их массы, но они почти отсутствуют в отобранных рентгеновских выборках ROSAT.\\

Для обнаружения кластеров SZ использовались три метода: две независимые реализации согласованного мультифильтра (MMF1 и MMF3) и PowellSnakes (PwS). Главный каталог построен как объединение 
каталогов трех методов. Полнота и надежность каталогов были оценены посредством внутренней и 
внешней проверки, как описано в разделе 4 справочного документа.\\

\section{ACT}
\cite{Act}
Это каталог из 4195 оптически подтвержденных скоплений галактик Сюняева-Зельдовича (SZ), 
обнаруженных на $13 168 deg^{2}$ неба (примерно 32\% всего неба), обследованных Космологическим 
телескопом Атакама (ACT). Кандидаты в кластеры были отобраны путем применения многочастотного
согласованного фильтра к картам 98 и 150 ГГц, построенным на основе всех наблюдений ACT, 
полученных в 2008–2018 гг., и впоследствии подтвержденных с помощью глубоких оптических обзоров с 
большой площадью. Обнаруженные кластеры охватывают диапазон красного смещения $0,04 < z <1,91$ со 
средним значением $z = 0,52$. Каталог содержит 221 кластер с $z > 1$, а всего 872 системы являются 
новыми открытиями. Выборка скоплений более чем в 22 раза больше, чем предыдущий каталог скоплений 
ACT, и на сегодняшний день является самой большой однородной выборкой скоплений, выбранных SZ. 
Зона обзора имеет большое перекрытие с глубокими оптическими исследованиями со слабым 
линзированием, которые используются для калибровки отношения масштабирования массы SZ-сигнала, 
такими как исследование темной энергии (Dark Energy Survey) ($4 552 deg^{2}$), стратегическая 
программа Hyper Suprime-Cam Subaru ($468 deg^{2}$) и Kilo Degree Survey ($823 deg^{2}$).\\


