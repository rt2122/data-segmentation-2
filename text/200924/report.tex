\documentclass{article}
    \usepackage{amssymb}
    \usepackage[utf8]{inputenc}
    \usepackage[russian]{babel}
    \usepackage[left=2cm,right=2cm,
        top=2cm,bottom=2cm,bindingoffset=0cm]{geometry}
    \usepackage{hyperref}
    \hypersetup{
        colorlinks=true,
        linkcolor=blue,
        filecolor=magenta,      
        urlcolor=cyan,
    }

\begin{document}
\begin{center}{\hugeОтчет по курсовой работе за неделю\\}\end{center}
Дата: 25.9.2020\\
Научные руководители: Герасимов С.В., Мещеряков А.В.\\
Студент: Немешаева Алиса\\
Курс: 4\\

\renewcommand{\labelitemi}{$\blacksquare$}
\renewcommand\labelitemii{$\square$}
\begin{enumerate}
    \item Главной целью на прошлую и текущую недели являлось повторение результатов \hyperlink{https://www.aanda.org/articles/aa/pdf/2020/02/aa36919-19.pdf}{статьи}
            (примерное совпадение графиков recall и false positive). В прошлый раз удалось достигнуть 
            recall 0,975 при пороговом значении для масок 0,1. В статье для такого порога значение 
            recall составило 1,0.\\
    \item Для достижения лучшего результата было обучено еще несколько моделей, их отличие от 
        предыдущей заключется в увеличении расстояния скоплений от центров патчей. Была достигнута 
        полнота 1.0 на тестовой области, но для более точного воспроизведения результатов нужно 
        продолжать обучение модели,
        \hyperlink{https://github.com/rt2122/data-segmentation-2/blob/master/Planck\_Unet/scan\_planck\_z\_f8\_d0.8.ipynb}{графики}.\\
\end{enumerate}

Общее количество строк кода: 1502\\
\end{document}
