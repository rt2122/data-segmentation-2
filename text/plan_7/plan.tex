\documentclass{article}
  \usepackage[utf8]{inputenc}
  \usepackage[russian]{babel}
  \usepackage[left=2cm,right=2cm,
    top=2cm,bottom=2cm,bindingoffset=0cm]{geometry}
  \usepackage{makecell}
 
\begin{document}
\begin{center}
{\huge План работы на 7 учебный семестр.}\\
\end{center}
\textit{Тема работы:}\\
Нейросетевые модели поиска и сегментации обьектов в данных современных космических обзоров (eRosita, ART-XC)\\
\textit{Научные руководители:}\\
Герасимов С.В., Мещеряков А.В.\\
\textit{Студент:}\\
Немешаева Алиса\\
    \begin{table}[h!]
        \begin{tabular}{|p{0.3\linewidth}|p{0.6\linewidth}|}
            \hline
            \textbf{Неделя} & \textbf{План работы на неделю}\\
            \hline
            18 сентября - 24 сентября & Воссоздание эксперимента с обучением Unet на данных 
                Planck. Обучение модели, создание алгоритма детекции, сравнение результатов.\\ 
            25 сентября - 1 октября & Дообучение модели, исправление ошибок. Изучение альтернативных 
                архитектур для нейросетевых моделей. \\
        \hline
            2 октября - 8 октября & Изучение микроволновых данных Atacama Cosmology Telescope. 
                Обучение модели. \\
            9 октября - 15 октября & Сравнение результатов новой модели с результами работы по 
                данным Planck.\\
            16 октября - 22 октября & Описание результатов в реферате.\\
        \hline
            23 октября - 29 октября & Алгоритм общей проекции данных Planck и ACT для создания 
                обучающих выборок.\\
            30 октября - 5 ноября & Обучение модели Unet на микроволновых данных из двух источников.\\
            6 ноября - 12 ноября & Подбор параметров для модели. \\
            13 ноября - 19 ноября & Описание результатов в реферате.\\
        \hline
            20 ноября - 26 ноября & Алгоритм проекции для оптических данных.\\
            27 ноября - 3 декабря & Обучение модели Unet на оптических данных.\\
            4 декабря - 10 декабря & Подбор параметров для нейросетевой модели. 
                Последние тестирования. Описание результатов в реферате.\\
            \hline
        \end{tabular}
    \end{table}
\end{document}
