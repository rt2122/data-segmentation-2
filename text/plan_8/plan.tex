\documentclass{article}
  \usepackage[utf8]{inputenc}
  \usepackage[russian]{babel}
  \usepackage[left=2cm,right=2cm,
    top=2cm,bottom=2cm,bindingoffset=0cm]{geometry}
  \usepackage{makecell}
 
\begin{document}
\begin{center}
{\huge План работы на 8 учебный семестр.}\\
\end{center}
\textit{Тема работы:}\\
Нейросетевые модели поиска и сегментации обьектов в данных современных космических обзоров (eRosita, ART-XC)\\
\textit{Научные руководители:}\\
Герасимов С.В., Мещеряков А.В.\\
\textit{Студент:}\\
Немешаева Алиса\\
    \begin{table}[h!]
        \begin{tabular}{|p{0.3\linewidth}|p{0.6\linewidth}|}
            \hline
            \textbf{Неделя} & \textbf{План работы на неделю}\\
            \hline
            18 февраля - 24 февраля & Построение актуальных графиков для статьи. Работа над 
                введением для статьи.\\ 
            25 февраля - 3 марта & Создание алгоритмов для обработки и подготовки рентгеновских 
                данных.\\
            \hline
            4 марта - 10 марта & Обучение нейросетевых моделей с рентгеновскими данными и 
                рентгеновскими скоплениями. Сравнение моделей, построение каталогов.\\
            11 марта - 17 марта & Обучение нейросетевых моделей с рентгеновскими данными и 
                скоплениями из разных диапазонов.\\
            18 марта - 24 марта & Создание алгоритмов для совмещения данных из разных диапазонов 
                (рентгеновский и микроволновой).\\
            25 марта - 31 марта & Обучение нейросетевых моделей с рентгеновскими и микроволновыми 
                данными на скоплениях из различных каталогов (рентгеновские и микроволновые).\\
            \hline
            1 апреля - 6 апреля & Описание результатов в тексте дипломной работы.\\
            7 апреля - 13 апреля & Обучение нейросетевых моделей с рентгеновскими и микроволновыми 
                данными на скоплениях из различных каталогов (скопления, найденные предыдущими 
                моделями).\\
            14 апреля - 20 апреля &Описание результатов в тексте дипломной работы.\\
            21 апреля - 27 апреля & Финальная работа над текстом дипломной работы.\\
            \hline
        \end{tabular}
    \end{table}
\end{document}
