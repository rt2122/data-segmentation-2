\documentclass{article}
    \usepackage{amssymb}
    \usepackage[utf8]{inputenc}
    \usepackage[russian]{babel}
    \usepackage[left=2cm,right=2cm,
        top=2cm,bottom=2cm,bindingoffset=0cm]{geometry}
    \usepackage{hyperref}
    \hypersetup{
        colorlinks=true,
        linkcolor=blue,
        filecolor=magenta,      
        urlcolor=cyan,
    }
  \usepackage{graphicx}
  \graphicspath{{pictures/}}
  \DeclareGraphicsExtensions{.pdf,.png,.jpg}

\begin{document}
\begin{center}{\hugeОтчет по курсовой работе за неделю\\}\end{center}
Дата: 22.10.2020\\
Научные руководители: Герасимов С.В., Мещеряков А.В.\\
Студент: Немешаева Алиса\\
Курс: 4\\

\renewcommand{\labelitemi}{$\blacksquare$}
\renewcommand\labelitemii{$\square$}
\begin{enumerate}
    \item На этой неделе основной целью было закрепление результатов прошлой недели, создание 
        каталога с наилучшими на данный момент параметрами и подготовка к созданию новой модели.\\
    \item Сгенерированы каталоги для thr = 0.1 и 0.2 (пороговое значение маски сегментации) и 
        step = 8 (шаг окна сканирования) для всего неба.\\
    \item Сгенерированы данные для обучения новой модели - теперь в данных для обучения будут 
        скопления из нового каталога ACT.\\
    \item Начато обучение новой модели.\\

\end{enumerate}

Отчет согласован с научным руководителем.\\
Общее количество строк кода за эту неделю: 108\\
\hyperlink{https://github.com/rt2122/data-segmentation-2}{Репозиторий}\\ 
\end{document}
