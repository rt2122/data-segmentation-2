\documentclass{article}
  \usepackage[utf8]{inputenc}
  \usepackage[russian]{babel}
  \usepackage[left=2cm,right=2cm,
    top=2cm,bottom=2cm,bindingoffset=0cm]{geometry}
 
\begin{document}
План работы на 7 учебный семестр.\\
Тема работы:\\
Нейросетевые модели поиска и сегментации обьектов в данных современных космических обзоров (eRosita, ART-XC)\\
Научные руководители:\\
Герасимов С.В., Мещеряков А.В.\\
Студент:\\
Немешаева Алиса\\
    \begin{table}[h!]
        \begin{tabular}{|p{0.2\linewidth}|p{0.7\linewidth}|}
            \hline
            \textbf{Неделя} & \textbf{План работы на неделю}\\
            \hline
            14 сентября - 20 сентября & Воссоздание эксперимента с обучением Unet на данных 
                Planck. Сравнение результатов для двух проекций - HEALPix и WCS.\\ 
            21 сентября - 27 сентября & Изучение влияния параметров архитекутры модели на результаты.\\
            28 сентября - 4 октября & Алгоритм проекции оптических данных DESI LIS для создания 
                обучающих выборок.\\
            5 октября - 11 октября & Обучение модели Unet на оптических данных. \\
            12 октября - 18 октября & Подбор параметров для модели.\\
            19 октября - 25 октября & Описание результатов в реферате.\\
            26 октября - 1 ноября & Алгоритм проекции рентгеновских данных для создания обучающих 
                выборок.\\
            2 ноября - 8 ноября & Обучение модели Unet на рентгеновских данных.\\
            9 ноября - 15 ноября & Подбор параметров для модели.\\
            16 ноября - 22 ноября & Описание результатов в реферате.\\
            23 ноября - 29 ноября & Алгоритм общей проекции для всех трёх видов данных.\\
            30 ноября - 6 декабря & Обучение модели Unet на всех данных.\\
            7 декабря - 13 декабря & Подбор параметров для нейросетевой модели.\\
            14 декабря - 20 декабря & Последние тестирования. Описание результатов в реферате.\\
            \hline
        \end{tabular}
    \end{table}
\end{document}
