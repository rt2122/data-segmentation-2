\documentclass{article}
    \usepackage{amssymb}
    \usepackage[utf8]{inputenc}
    \usepackage[russian]{babel}
    \usepackage[left=2cm,right=2cm,
        top=2cm,bottom=2cm,bindingoffset=0cm]{geometry}
    \usepackage{hyperref}
    \hypersetup{
        colorlinks=true,
        linkcolor=blue,
        filecolor=magenta,      
        urlcolor=cyan,
    }

\begin{document}
\begin{center}{\hugeОтчет по курсовой работе за неделю\\}\end{center}
Дата: 17.9.2020\\
Научные руководители: Герасимов С.В., Мещеряков А.В.\\
Студент: Немешаева Алиса\\
Курс: 4\\

\renewcommand{\labelitemi}{$\blacksquare$}
\renewcommand\labelitemii{$\square$}
За прошедшую неделю была проведена следующая работа:\\
\begin{enumerate}
    \item Построена \hyperlink{https://github.com/rt2122/data-segmentation-2/blob/master/Planck\_Unet/train\_unet\_planck\_z\_8\_filters.ipynb}{модель}, повторяющая по параметрам модель из \hyperlink{https://www.aanda.org/articles/aa/pdf/2020/02/aa36919-19.pdf}{статьи}, а также другая \hyperlink{https://github.com/rt2122/data-segmentation-2/blob/master/Planck\_Unet/train\_unet\_planck\_z.ipynb}{модель}, с несколько другими параметрами (в два раза больше количество фильтров и на два слоя Dropout меньше для каждого блока декодера).\\
    \item Проведены \hyperlink{https://github.com/rt2122/data-segmentation-2/blob/master/Planck\_Unet/scan\_planck\_z\_f8.ipynb}{проверки} модели на тестовой области, полнота на которой достигла 0,975.\\
    \item Создан \hyperlink{https://github.com/rt2122/data-segmentation-2/blob/master/modules/DS\_detector.py}{более быстрый метод} сканирования тестовой области.\\

\end{enumerate}

Общее количество строк кода: 1437\\
\end{document}
