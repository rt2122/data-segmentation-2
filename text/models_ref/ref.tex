\documentclass{article}
    \usepackage{amssymb}
    \usepackage[utf8]{inputenc}
    \usepackage[russian]{babel}
    \usepackage[left=2cm,right=2cm,
        top=2cm,bottom=2cm,bindingoffset=0cm]{geometry}
    \usepackage{hyperref}
    \hypersetup{
        colorlinks=true,
        linkcolor=blue,
        filecolor=magenta,      
        urlcolor=cyan,
    }
  \usepackage{graphicx}
  \graphicspath{{pictures/}}
  \DeclareGraphicsExtensions{.pdf,.png,.jpg}

\begin{document}
\begin{center}{\huge Сравнение нескольких нейросетевых архитектур\\}\end{center}
\section{PSPNet}
Эта модель была создана для использования в области scene parsing. Главная задача, которую эта 
модель помогает решить - попиксельная сегментация объектов на изображении при условии наличия 
большого количества меток (например, датасет ADE20K, о котором идёт речь в статье, содержит 
изображения с 150 метками). \\

PSPNet помогает решить проблемы:\\
\begin{itemize}
    \item взаимосвязи меток (например, объект 
        с меткой <<самолёт>> скорее всего будет находиться в пространстве с меткой <<аэропорт>> или 
        <<посадочная полоса>>)\\
    \item совпадающих категорий (наличие в тренировочной выборке объектов вроде <<поле>> и <<земля>>, 
        <<холм>> и <<гора>>)\\
    \item небольших объектов (например, <<фонарь>> или <<вывеска>>, находящиеся на дальнем плане 
        изображения)\\
\end{itemize}

Из вышеперечисленных проблем в текущей работе по сегментации скоплений нас может интересовать только 
последняя - искомое скопление может занимать маленькую площадь относительно всего изображения.\\

Основная идея PSPNet заключается в использовании Pyramid Pooling Module. Чтобы получить этот модуль, 
нужно сначала сделать несколько версий изначального изображения в разных масштабах с помощью pooling 
слоёв разных размеров, первый <<грубый>> слой собирает данные всего изображения в один многоканальный 
пиксель, следующий содержит данные нескольких подрегионов, и далее каждый последующий содержат всё 
меньше глобальной информации и всё больше локальной. \\

Далее для каждого масштаба проводится свёртка для уменьшения количества каналов и upscaling с помощью 
билинейной интерполяции, и все масштабы конкатенируются, чтобы с помощью последнего слоя свёртки 
получить для них итоговую сегментацию. \\

\includegraphics[width=\linewidth]{ppm}

\section{LinkNet}
Эта модель решает проблему большого количества параметров и низкой скорости других моделей из 
похожей области применения. LinkNet хорошо подойдёт для сегментации в реальном времени и сегментации 
на видео, что тоже имеет очень слабое отношение к нашей области, где количество данных фиксировано и 
сегментация в реальном времени не требуется, однако скорость нейросетевой модели тем не менее является 
плюсом.\\

Общая структура LinkNet очень напоминает UNet, с тем отличием, что в блоках кодировщика добавлены 
дополнительные skip-connection связи.\\


\includegraphics[width=0.3\linewidth]{linknet}
\includegraphics[width=0.3\linewidth]{link_encoder}
\includegraphics[width=0.3\linewidth]{link_decoder}\\

\section{W-Net}
Эта архитектура была призвана улучшить результаты сегментации при условиях:\\

\begin{itemize}
    \item без использования сложных архитектур свёрточных нейронных сетей\\ 
    \item при тестировании на данных из разных датасетов\\
\end{itemize}

При сегментации космических данных мы обычно имеем общий обзор некоторой области неба в 
определенном спектре, и эти данные получены с помощью определенного телескопа. Однако возможно 
появится причина обучать модель на данных одного телескопа, а тестировать на других, что в какой-то 
степени кореллирует с описанными условиями.\\

Архитектура W-Net предлагает улучшение модели U-Net: обучение происходит на двух последовательных 
нейросетевых моделях. Выход первой модели конкатенируется с входом и отправляется на обработку второй 
моделью.\\

\includegraphics[width=\linewidth]{wnet}\\

\end{document}
