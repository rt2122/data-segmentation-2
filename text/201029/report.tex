\documentclass{article}
    \usepackage{amssymb}
    \usepackage[utf8]{inputenc}
    \usepackage[russian]{babel}
    \usepackage[left=2cm,right=2cm,
        top=2cm,bottom=2cm,bindingoffset=0cm]{geometry}
    \usepackage{hyperref}
    \hypersetup{
        colorlinks=true,
        linkcolor=blue,
        filecolor=magenta,      
        urlcolor=cyan,
    }
  \usepackage{graphicx}
  \graphicspath{{pictures/}}
  \DeclareGraphicsExtensions{.pdf,.png,.jpg}

\begin{document}
\begin{center}{\hugeОтчет по курсовой работе за неделю\\}\end{center}
Дата: 29.10.2020\\
Научные руководители: Герасимов С.В., Мещеряков А.В.\\
Студент: Немешаева Алиса\\
Курс: 4\\

\renewcommand{\labelitemi}{$\blacksquare$}
\renewcommand\labelitemii{$\square$}
\begin{enumerate}
    \item Найдена ошибка в генерации данных: в 37-ом пикселе из разбиения $n_{nside}=2$ healpix 
        содержался участок с очень большим по модулю значением, что приводило в определённых 
        ситуациях к превращения значения функции потерь в $NaN$.\\
    \item Создан новый алгоритм генерации данных для обучения - предыдущий использовал заранее 
        сгенерированные патчи, из-за чего использовал слишком много пространства на диске, новый 
        алгоритм генерирует данные прямо перед началом обучения, и требует почти в 10 раз меньше 
        данных, хранящихся на диске, что позволяет впоследствии перенести обучение на удалённые 
        сервера.\\
\end{enumerate}

Отчет согласован с научным руководителем.\\
Общее количество строк кода за эту неделю: \\
\hyperlink{https://github.com/rt2122/data-segmentation-2}{Репозиторий}\\ 
\end{document}
