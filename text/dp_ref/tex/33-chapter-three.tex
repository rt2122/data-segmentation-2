\chapter{Научная задача в рамках спецсеминара}
\label{cha:ch_3}

Моя работа в рамках спецсеминара рассматривает возможность применения нейросетевых методов к 
решению проблемы сегментации и детекции объектов по многоволновым данным космических телескопов 
(в данном случае оптического, микроволнового и рентгеновского диапазонов). В качестве основы для нейросетевой архитектуры использовалась модель U-net.\\

Вся программная часть научной работы ведётся на языке Python, так как его библиотеки лучше всего 
подходят для обработки научных данных. Кроме того, этот язык является самым популярным инструментом 
в области глубокого обучения, и для этих целей также существует большое количество библиотек для 
разных случаев.\\

Обработка космических данных подразумеват наличие возможности добавления иллюстраций совместно с 
кодом, проведение экспериментов и сравнение их между собой. Лучше всего для такого формата работы 
подойдёт Jupyter Notebook. <<Ноутбуками>> удобно обмениваться, изменять и редактировать отдельные 
данные, проверяя эксперимент в одельной <<клетке>>, при этом не нужно перезапускать всю программу 
заново. Для обучения нейронных сетей формат <<ноутбуков>> тоже очень хорошо подходит - вместе с 
кодом программы сохраняется информация о том, как проходило обучение модели. \\

Однако кроме проведения экспериментов и обучения нейросетевых моделей иногда приходится 
программировать модули с функциями, которые так или иначе понадобятся ещё не раз - и лучше всего 
здесь справляется Vim. С его помощью можно быстро открыть файл нужного модуля, найти функцию и 
отредактировать её параметры. Однако для этого можно использовать любую IDE, но Vim окажется 
быстрее большинства таких вариантов.\\
