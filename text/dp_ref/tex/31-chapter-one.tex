\chapter{История появления интегрированных сред разработки ПО}
\label{cha:ch_1}
Интегрированная среда разработки, ИСР (англ. IDE, Integrated Development Environment или Integrated Debugging Environment) — объединение программных средств, используемое разработчиками для создания программного обеспечения (ПО).\\
Классическая среда разработки включает в себя:\\
\begin{itemize}
    \item текстовый редактор;\\
    \item компилятор и / или интерпретатор;\\
    \item средства автоматизации сборки;\\
    \item отладчик.\\
\end{itemize}
Иногда IDE может содержать также средства для интеграции с системами управления версиями и различные инструменты для упрощения конструирования графического интерфейса для конечного пользователя. 
Современные среды разработки часто также включают браузер классов, инспектор объектов и диаграмму иерархии классов — для упрощения систематизации при объектно-ориентированной разработке ПО. 
Хотя и существуют IDE, предназначенные для нескольких языков программирования — такие, как Eclipse, NetBeans, Embarcadero RAD Studio, Qt Creator или Microsoft Visual Studio, но чаще всего IDE предназначается для одного определённого языка программирования - как, например, Visual Basic, PureBasic, Delphi, Dev-C++.\\

Первые IDE были созданы для работы через консоль или терминал, которые сами по себе только недавно вошли в употребление на тот момент времени: до того программы создавались на бумаге, вводились в машину с помощью предварительно подготовленных бумажных носителей (перфокарт, перфолент) и так далее.\\

Dartmouth BASIC был первым языком, который был создан с IDE, и был также первым языком, что был разработан для использования в консоли или терминале. Эта IDE (часть Dartmouth Time Sharing System) управлялась при помощи команд, поэтому существенно отличалась от более поздних, управляемых с помощью меню и горячих клавиш, и тем более графических IDE, распространённых в XXI веке. Однако она позволяла править исходный код, управлять файлами, компилировать, отлаживать и выполнять программы способом, принципиально подобным современным IDE.\\

Maestro I — продукт от Softlab Munich, был первой в мире интегрированной средой разработки для программного обеспечения в 1975 г. и, возможно, мировым лидером в этой рыночной нише в течение 1970-х и 1980-х годов. Он был установлен у 22000 программистов во всем мире. До 1989 года 6000 копий было установлено в Федеративной Республике Германия. Ныне Maestro I принадлежит истории и может быть найден разве что в Музее Информационной технологии в Арлингтоне.\\

Одной из первых IDE с возможностью подключения плагинов была Softbench.\\

\newpage
\begin{table}[h]
    \caption{Несколько популярных IDE по годам их появления}
    \centering
    \begin{tabular}[h!]{| p{0.055\linewidth} | p{0.25\linewidth} | p{0.55\linewidth} |}
        \hline
        \rowcolor{gray}Год & Название IDE & Разработчик\\\hline\hline
        1976 & Emacs & David A. Moon, Guy L. Steele Jr.\\\hline
        1991 & Vim & Bram Moolenaar\\\hline
        1997 & Visual Studio & Microsoft\\\hline
        1997 & NetBeans & Apache Software Foundation, Oracle, Sun Microsystems\\ \hline
        1999 & KDevelop & KDE\\\hline
        2001 & Eclipse & IBM, Eclipse Foundation\\\hline
        2001 & IntelliJ IDEA & JetBrains\\\hline
        2003 & XCode & Apple Inc.\\\hline
        2005 & Code::Blocks & The Code::Blocks team\\\hline
        2005 & Oracle Solaris Studio & Oracle Corporation \\\hline 
        2006 & CodeLite & Eran Ifrah \\\hline
        2007 & Qt Creator & Qt Project\\\hline
        2007 & Komodo & ActiveState \\\hline
        2008 & Sublime Text & Jon Skinner\\\hline
        2009 & PhpStorm & JetBrains\\\hline
        2009 & Spyder & Pierre Raybaut \\\hline
        2013 & Xamarin & Microsoft\\\hline
        2014 & Atom & GitHub Inc. \\\hline
        2014 & Project Jupyter & Project Jupyter\\\hline
        2015 & Visual Studio Code & Microsoft \\\hline
    \end{tabular}
\end{table}
