\Conclusion % заключение к отчёту

По информации выше можно судить, что для всех популярных языков программирования существует 
огромное множество сред разработки. В общем смысле они имеют одни и те же базовые функции, и выбор 
пользователя уже зависит от ограничений отдельных IDE. \\

Для кого-то важнее всего будет возможность 
настроить абсолютно любую деталь интерфейса, для кого-то принципиальна совместимость с наибольшим 
количеством операционных систем (если, например, разработчик по каким-то причинам пользуется 
средой разработки на нескольких устройствах с разными операционными системами), для кого-то важна 
возможность добавить определённый компилятор или отладчик для работы, для кого-то важна скорость 
загрузки IDE и её легковесность (если, например нужно работать на устройстве с ограниченными 
характеристиками).\\

Идеальным вариантом является Vim, как инструмент обладающий абсолютно всеми возможными свойствами, 
однако для его настройки требуется большое количество времени, и ещё больше времени нужно 
пользователю для того, чтобы запомнить все нужные команды и в принципе освоиться.\\

Так или иначе выбор IDE будет зависеть от конкретной ситуации и от разрабатываемого продукта. Лучше
всего, когда у разработчика есть выбор, и он имеет возможность использовать любимую IDE - это может 
положительно сказаться на качестве продукта. Однако при работе в команде приходиься учитывать 
интересы других разработчиков и общие правила компании. В любом случае, после определенного 
количества времени, проведённого за работой в какой-либо IDE, программист привыкает к ней, и 
скорость и качество работы уже не будут зависеть от этого фактора.\\
