\chapter{Обзор современных сред разработки}
\label{cha:ch_2}

Интегрированная среда разработки (IDE) - это своего рода расширенный редактор для одного или нескольких 
языков программирования со встроенным компилятором, окном редактирования и утилитами отслеживания или
отладки ошибок, которые упрощают разработку программы. IDE облатает кроме того ещё многими другими 
функциями, например:\\
\begin{itemize}
    \item Пользователь может просматривать файлы через IDE, чтобы найти тот, который ему нужен;\\
    \item Доступна <<справка>>, с помощью которой пользователь может быстро узнать, какие аргументы 
        требуются конкретному методу или что может означать ошибка;\\
    \item Пользователь может видеть измененные значения переменных, объектов и функций в режиме 
        реального времени;\\
\end{itemize}

IDE - это программы для ввода исходного кода. Обычно это среды редактирования с инструментами, 
помогающими программистам быстро и эффективно писать исходный код. Например пользователь может 
создавать веб-приложения, управляемые PHP, используя комбинацию Eclipse и PHP Eclipse.\\

Основные функции IDE обычно включают:\\
\begin{itemize}
    \item Анализ кода: это возможность среды IDE знать ключевые слова и функции языка. IDE может
        использовать эту информацию, чтобы исполнять действия, как:\\
        \begin{itemize}
            \item выделение типографические ошибок;\\
            \item вывод списка всех доступных функций на основе соответствующей ситуации;\\
            \item предположение определения функции;\\
            \item подсветка различных элементов кода разными цветами для различных ключевых слов 
                и функций.\\
        \end{itemize}
    \item Управление ресурсами: при создании приложений языки часто полагаются на то, что 
        определенные ресурсы, такие как библиотека или файлы заголовков, находятся в определенном 
        каталоге. IDE управляют этими ресурсами. IDE знает необходимые пути к библиотекам и 
        заголовочным файлам, поэтому ошибки можно обнаружить на этапе разработки, на этапе 
        компиляции или сборки.\\
    \item Инструменты отладки. В среде IDE пользователю предоставляется возможность тщательно
        протестировать свое приложение перед выпуском. IDE может предоставлять значения переменных 
        в определенные моменты, подключаться к разным репозиториям данных или принимать разные 
        параметры времени выполнения.\\
    \item Компиляция и сборка: для языка, которому требуется этап компиляции или сборки, IDE 
        переводит код с языка высокого уровня в объектный код целевой платформы. Требования к этой 
        функции существенно различаются от языка к языку.\\
\end{itemize}

Таким образом, традиционно IDE специализируется на одном языке или наборе подобных языков. 
Некоторые известные IDE и их языки включают: JBuilder для Java; Пакет Metrowerks CodeWarrior для 
Java, C и C ++; и Microsoft Visual Studio для семейства языков Visual Basic и C#.
Далее будет проведено сравнение некоторых существующих IDE по их характеристикам.\\

\section{IntelliJ Idea}
IntelliJ IDEA - это интегрированная среда разработки (IDE), написанная на Java для разработки 
компьютерного программного обеспечения. Он разработана компанией JetBrains (ранее известной как 
IntelliJ).\\

Эта IDE предоставляет определенные функции, такие как завершение кода путем анализа контекста, 
навигация по коду, которая позволяет напрямую переходить к классу или объявлению в коде, 
рефакторинг кода, отладку кода, линтинг и варианты исправления несоответствий с помощью предложений.
IDE обеспечивает интеграцию с инструментами сборки / упаковки, такими как grunt, bower, gradle и 
SBT. Он поддерживает системы контроля версий, такие как Git, Mercurial, Perforce и SVN. К базам 
данных, таким как Microsoft SQL Server, Oracle, PostgreSQL, SQLite и MySQL, можно получить доступ
непосредственно из среды IDE в версии Ultimate через встроенную версию DataGrip.\\

IntelliJ поддерживает плагины, с помощью которых можно добавлять в IDE дополнительные функции. 
Плагины можно загружать и устанавливать либо с веб-сайта репозитория плагинов IntelliJ, либо с 
помощью встроенной в IDE функции поиска и установки плагинов. Каждая редакция имеет отдельные 
репозитории плагинов, причем в редакциях Community и Ultimate насчитывается более 3000 плагинов 
каждая по состоянию на 2019 год.\\

\section{Eclipse}
Eclipse написана на языке Java. Eclipse - это многоязычная среда разработки программного 
обеспечения, состоящая из интегрированной среды разработки (IDE) и расширяемой системы подключаемых
модулей. Eclipse SDK (Software Development Kit) включает Eclipse Java Development Tools (JDT), 
предлагающий IDE со встроенным инкрементным компилятором Java и полную модель исходных файлов Java. 
Это позволяет применять передовые методы рефакторинга и анализа кода. IDE также использует рабочее
пространство. Eclipse реализует виджеты с помощью набора инструментов виджетов для Java, называемого 
SWT, в отличие от большинства приложений Java, которые используют стандартный набор инструментов
абстрактного окна Java (AWT) или Swing. Пользовательский интерфейс Eclipse также использует 
промежуточный уровень графического интерфейса, называемый JFace, который упрощает создание 
приложений на основе SWT.\\

Eclipse - это расширяемая платформа для создания IDE. Она предоставляет ядро служб для управления 
набором инструментов, работающих вместе для поддержки задач программирования. Создатели 
инструментов вносят свой вклад в платформу Eclipse, упаковывая свои инструменты в подключаемые 
компоненты, называемые подключаемыми модулями Eclipse, которые соответствуют контракту подключаемых 
модулей Eclipse. \\

\section{Qt Creator}
Qt Creator была разработана Qt Development Frameworks на C++. Это кроссплатформенная IDE,
работающая только для языка C++. Она включает в себя визуальный отладчик, интегрированный 
графический интерфейс и конструктор форм. Возможности IDE включают выделение синтаксиса, 
автодополнение, редактор для написания правильно отформатированного кода, конструктор 
пользовательского интерфейса и так далее. Он предоставляет плагин отладчика, который действует как
интерфейс между ядром Qt Creator и внешним собственным отладчиком для отладки языка C++. Qt Creator
обеспечивает поддержку для создания и запуска приложений Qt для сред персональных компьютеров 
(Windows, Linux, FreeBSD и Mac OS) и мобильных устройств (Symbian, Maemo и MeeGo).\\

\section{PyCharm}
PyCharm - это интегрированная среда разработки, используемая в компьютерном программировании, 
особенно для языка Python. Она также разработана чешской компанией JetBrains. Она обеспечивает 
анализ кода, графический отладчик, интегрированный тестер модулей, интеграцию с системами контроля 
версий (VCSes) и поддерживает веб-разработку с помощью Django, а также анализ данных с помощью 
Anaconda.\\

PyCharm является кросс-платформенной IDE, с версиями для Windows, macOS и Linux. Community Edition
выпускается под лицензией Apache License, а также существует Professional Edition с дополнительными
функциями, выпущенная под частной лицензией.\\

\section{Jupyter Notebook}
Jupyter Notebook (ранее IPython Notebooks) - это интерактивная вычислительная среда в виде 
веб-приложения для создания документов Jupyter notebook. Термин «notebook» может в разговорной речи
относиться к множеству различных сущностей, в основном к веб-приложению Jupyter Notebook, 
веб-серверу Jupyter Python или формату документа Jupyter в зависимости от контекста. Документ 
Jupyter Notebook - это документ JSON, соответствующий схеме с контролем версий и содержащий 
упорядоченный список ячеек ввода / вывода, который может содержать код, текст (с использованием 
языка разметки Markdown), математику, графики и мультимедийные данные, обычно заканчивающиеся 
расширением ".ipynb".\\

Jupyter Notebook можно преобразовать в несколько открытых стандартных форматов вывода (HTML, 
слайды презентации, LaTeX, PDF, ReStructuredText, Markdown, Python) с помощью функции 
«Загрузить как» в веб-интерфейсе, с помощью библиотеки nbconvert или команды «jupyter nbconvert». 
Чтобы упростить визуализацию документов Jupyter Notebook в Интернете, библиотека nbconvert 
предоставляется как услуга через NbViewer, которая может принимать URL-адрес любого общедоступного
документа Jupyter Notebook, на лету конвертировать его в HTML и отображать его пользователю.\\

Jupyter Notebook может подключаться ко многим ядрам, что позволяет программировать на многих языках. 
По умолчанию Jupyter Notebook поставляется с ядром IPython. На момент выпуска версии 2.3 
(октябрь 2014 г.) существует 49 Jupyter-совместимых ядер для многих языков программирования, включая
Python, R, Julia и Haskell.\\

Интерфейс Notebook был добавлен в IPython в версии 0.12 (декабрь 2011 г.), а в 2015 г. переименован 
в Jupyter notebook (IPython 4.0 - Jupyter 1.0). Jupyter Notebook похож на интерфейс других программ, 
таких как Maple, Mathematica и SageMath, стиль вычислительного интерфейса, зародившийся в Mathematica 
в 1980-х годах. По данным The Atlantic, в начале 2018 года интерес к Jupyter обогнал популярность
интерфейса ноутбука Mathematica.\\

\section{Microsoft Visual Studio}
Microsoft Visual Studio - это интегрированная среда разработки (IDE) от Microsoft. Она используется 
для разработки компьютерных программ, а также веб-сайтов, веб-приложений, веб-сервисов и мобильных
приложений. Visual Studio использует платформы разработки программного обеспечения Microsoft, такие 
как Windows API, Windows Forms, Windows Presentation Foundation, Windows Store и Microsoft 
Silverlight.\\

Visual Studio включает редактор кода, поддерживающий IntelliSense (компонент автодополнения кода), а 
также инструмент рефакторинга кода. Интегрированный отладчик работает как отладчик на уровне 
исходного кода, так и как отладчик на уровне компьютера. Другие встроенные инструменты включают
профилировщик кода, конструктор для создания приложений с графическим интерфейсом, веб-дизайнер,
конструктор классов и конструктор схемы базы данных. Он принимает плагины, которые расширяют
функциональность почти на всех уровнях, включая добавление поддержки систем управления версиями 
(таких как Subversion и Git) и добавление новых наборов инструментов, таких как редакторы и 
визуальные дизайнеры, для языков, специфичных для предметной области, или наборов инструментов для 
других аспектов разработки программного обеспечения.\\

Visual Studio поддерживает 36 различных языков программирования и позволяет редактору кода и 
отладчику поддерживать (в разной степени) практически любой язык программирования при условии, 
что существует служба для конкретного языка. Встроенные языки включают C, C ++, C ++ / CLI, Visual 
Basic .NET, C #, F #, JavaScript, TypeScript, XML, XSLT, HTML и CSS. Поддержка других языков, 
таких как Python, Ruby, Node.js и M среди других, доступна через плагины. Java (и J #) 
поддерживались в прошлом.\\

Самая базовая версия Visual Studio, версия Community, доступна бесплатно. Слоган Visual Studio 
Community edition: «Бесплатная полнофункциональная IDE для студентов, разработчиков с открытым 
исходным кодом и индивидуальных разработчиков».\\

\section{Code::Blocks}
Code::Blocks разработанa командой разработчиков Code::Blocks team. Написана на C++. Онa использует
кроссплатформенную операционную систему. Это бесплатная IDE с открытым исходным кодом. Eё можно
использовать как для языка C, так и для C++. Code::Blocks поддерживает несколько компиляторов, 
включая MinGW / gcc, Digital Mars, Microsoft Visual C ++, Watcom, LCC, Borland C ++ и компилятор 
Intel C ++ и так далее.\\ 

Функции IDE включают:\\ 
\begin{itemize}
    \item выделение синтаксиса;\\
    \item сворачивание кода с помощью компонента редактора Scintilla, C++;\\ 
    \item автозавершение кода; \\
    \item обозреватель классов;\\ 
    \item интегрированный список задач;\\ 
    \item интегрированный интерфейс отладчика, который поддерживает GDB (GNU Debugger) и, в 
        некоторой степени, консольный отладчик Microsoft-CDB (Console Debugger);\\ 
\end{itemize}
Для набора инструментов wxWidgets существует интегрированный плагин быстрой разработки приложений 
под названием wxSmith.\\

\section{KDevelop}
KDevelop разработан KDE (K Desktop Environment) и полезен в системе Linux, чтобы предоставить 
пользователю систему, похожую внешне на Windows. Написана на C++. Она использует кроссплатформенную
операционную систему и работает на платформе KDE. Это бесплатное программное обеспечение. KDE не 
включает в состав KDevelop компилятор. Вместо этого используется внешний компилятор, такой как 
gcc/g++, для создания исполняемого кода. KDevelop использует встроенный текстовый редактор, а 
редактором по умолчанию является KDE Advanced Text Editor. Этот редактор по умолчанию можно 
заменить редактором на основе Qt. Он включает в себя такие функции, как редактор исходного кода с
подсветкой синтаксиса и автоматическим отступом, браузер классов, конструктор графического 
интерфейса пользователя, интерфейс для коллекции компиляторов GNU и отладчик GNU, мастера для 
создания и обновления определений классов и инфраструктуры приложения, автоматическое завершение 
кода, встроенное в поддержке Doxygen. Он использует архитектуру на основе плагинов. Это не зависит 
от языка программирования и системы. Он поддерживает другие технологии, такие как Qt, GTK + и 
wxWidgets.\\

\section{CodeLite}
CodeLite разработана Eran Ifrah на C ++. Использует кроссплатформенную операционную систему. Это 
бесплатная IDE с открытым исходным кодом для языков C и C ++. Для программирования используется 
набор инструментов wxWidget. Программа компилируется и отлаживается с помощью бесплатных 
инструментов (MinGW и GDB) для Mac OS X, Windows, Linux и FreeBSD. Она запускает любой сторонний 
компилятор или инструмент с интерфейсом командной строки. В его функции входят управление проектами
(рабочая область / проекты), завершение кода, рефакторинг кода, просмотр исходного кода, выделение
синтаксиса, интеграция с Subversion, интеграция cscope, интеграция Unittest ++, интерактивный 
отладчик, построенный на GDB, и редактор исходного кода.\\

\section{Dev-C++}
Dev-C ++ разработана Bloodshed Software на Delphi. Она была создана для операционной системы 
Microsoft Windows. Dev-C++ - это бесплатная IDE, распространяемая под Стандартной 
общественной лицензией GNU для программирования на C и C++. Она поставляется в комплекте с MinGW,
бесплатным компилятором. Хостинг проекта - SourceForge. Dev-C++ была первоначально разработана
программистом Колином Лапласом. Dev-C++ также можно использовать в сочетании с Cygwin или любым 
другим компилятором на основе gcc. Еще один аспект Dev-C++ - это использование DevPaks, упакованных
расширений в среде программирования с дополнительными библиотеками, шаблонами и утилитами. Пакеты 
DevPak часто содержат, (но не ограничиваются ими), утилиты с графическим интерфейсом, в том числе
популярные наборы инструментов, такие как GTK+, wxWidgets и FLTK. Пользователи Dev-C++ могут 
загружать дополнительные библиотеки или пакеты кода, которые увеличивают объем и функциональность 
Dev-C++, например графику, сжатие, анимацию, поддержку звука и многое другое.\\

\section{BlueJ}
BlueJ разработана BlueJ team. Первоначальный автор - Майкл Коллинг. Эта IDE написан на языке Java. 
Она использует кроссплатформенную операционную систему и платформу Java. BlueJ разработана в 
основном для образовательных целей, но подходит для разработки маломасштабного программного 
обеспечения. BlueJ была создана для поддержки преподавания и изучения объектно-ориентированного
программирования. Её функции включают представление объектной ориентации, простоту интерфейса,
взаимодействие с объектами, панель кода, регрессионное тестирование, поддержку групповой работы, 
гибкую систему расширений, файлы Jar и апплеты, переводы и т. д. Работа в среде BlueJ позволяет
получить конкретный опыт использования таких абстрактных концепции, как отношения класса / объекта,
создание экземпляра объекта, вызов метода, передача параметров и т. д. Эти абстрактные концепции
традиционно трудны для понимания новичком, и предоставление конкретных представлений для них 
предназначено для облегчения процесса обучения. Хотя это эффективное программное обеспечение, в нем отсутствуют некоторые функции, такие как проверка кода в реальном времени и обнаружение ошибок,
предлагаемые исправления для предупреждений / ошибок, сворачивание кода и т. д.\\

\section{NetBeans}
NetBeans - продукт корпорации Oracle. Эта IDE написана на Java для кроссплатформенной операционной 
системы. Она использует платформу Java SE для исполнения программ. IDE NetBeans работает везде, 
где установлена JVM, включая Windows, Mac OS, Linux и Solaris. Платформа NetBeans позволяет 
разрабатывать приложения из набора составных программных компонентов, называемых модулями. 
Приложения, основанные на платформе NetBeans (включая IDE NetBeans), могут быть расширены 
сторонними разработчиками. IDE NetBeans - это интегрированная среда разработки с открытым исходным 
кодом. IDE NetBeans поддерживает разработку всех типов приложений Java, а именно Java SE, включая 
JavaFX, Java ME, EJB и т. д. Поддерживает функцию модульности. Каждый модуль предоставляет четко
определенные функции, такие как поддержка языка Java, редактирование или поддержка CVS (система
одновременного управления версиями) и SVN (дополнительный номер версии). NetBeans также включает 
профилировщик, который помогает разработчикам находить утечки памяти и оптимизировать скорость. 
Инструменты проектирования графического интерфейса пользователя позволяют создавать прототипы и
проектировать графические интерфейсы пользователя Swing путем перетаскивания и размещения 
компонентов графического интерфейса. Редактор JavaScript обеспечивает расширенную поддержку 
JavaScript, Ajax и CSS. Функции редактора JavaScript включают выделение синтаксиса, рефакторинг, 
завершение кода для собственных объектов и функций, создание скелетов классов JavaScript, 
создание обратных вызовов Ajax из шаблона и автоматические проверки совместимости браузера.\\

\newpage
\section{Сравнение IDE для языков C/C++}
\begin{table}[h!]
    \caption{Сравнение IDE для языков C/C++}
    \centering
    \begin{tabular}[h!]{| p{0.15\linewidth} | p{0.2\linewidth} | p{0.1\linewidth} | p{0.1\linewidth} | p{0.1\linewidth} | p{0.1\linewidth} | p{0.1\linewidth} |}
        \hline
        \rowcolor{gray} IDE & Лицензия &Windows & Linux & Отладчик & Автодо-\linebreakполнение & Браузер классов\\\hline\hline
        Code::Blocks  & GPL & Да \cellcolor{Red}   & Да \cellcolor{Red}   & Да \cellcolor{Red}   & Да \cellcolor{Red}   & Да \cellcolor{Red}  \\\hline
        CodeLite  & GPL & Да \cellcolor{Red}   & Да \cellcolor{Red}   & Да \cellcolor{Red}   & Да \cellcolor{Red}   & Да \cellcolor{Red}  \\\hline
        Dev-C++  & GPL & Да \cellcolor{Red}   & Нет \cellcolor{green}   & Да \cellcolor{Red}   & Нет \cellcolor{green}   & Нет \cellcolor{green}  \\\hline
        Eclipse  &EPL & Да \cellcolor{Red}   & Да \cellcolor{Red}   & Да \cellcolor{Red}   & Да \cellcolor{Red}   & Нет \cellcolor{green}  \\\hline 
        KDevelop  & GPL & Да \cellcolor{Red}   & Да \cellcolor{Red}   & Да \cellcolor{Red}   & Да \cellcolor{Red}   & Да \cellcolor{Red}  \\\hline
        NetBeans  & CDDL, GPL, LGPL & Да \cellcolor{Red}   & Да \cellcolor{Red}   & Да \cellcolor{Red}   & Да \cellcolor{Red}   & Да \cellcolor{Red}  \\\hline
        Qt Creator  &GPL& Да \cellcolor{Red}   & Да \cellcolor{Red}   & Да \cellcolor{Red}   & Да \cellcolor{Red}   & Да \cellcolor{Red}  \\\hline
        Visual Studio  & Проприетарная & Да \cellcolor{Red}   & Нет \cellcolor{green}   & Да \cellcolor{Red}   & Да \cellcolor{Red}   & Да \cellcolor{Red} \\\hline
    \end{tabular}
\end{table}

\section{Сравнение IDE для Python}
\begin{table}[h!]
    \caption{Сравнение IDE для Python}
    \centering
    \begin{tabular}[h!]{| p{0.2\linewidth} | p{0.2\linewidth} | p{0.13\linewidth} | p{0.13\linewidth} |} 
        \hline
        \rowcolor{gray} IDE & Лицензия & Windows & Linux \\\hline\hline
        NetBeans & CDDL, GPL, LGPL  & Да \cellcolor{Red}   & Да \cellcolor{Red}  \\\hline
        PyCharm & ASL  & Да \cellcolor{Red}   & Да \cellcolor{Red} \\\hline
        Spyder & MIT  & Да \cellcolor{Red}   & Да \cellcolor{Red}  \\\hline
        Eclipse & EPL  & Да \cellcolor{Red}   & Да \cellcolor{Red}  \\\hline
        Visual Studio & Проприетарная  & Да \cellcolor{Red}   & Нет \cellcolor{green} \\\hline
    \end{tabular}
\end{table}

\section{Сравнение IDE для Java}
\begin{table}[h!]
    \caption{Сравнение IDE для Java}
    \centering
    \begin{tabular}[h!]{| p{0.2\linewidth} | p{0.2\linewidth} | p{0.13\linewidth} | p{0.13\linewidth} | p{0.13\linewidth} |} 
        \hline
    \rowcolor{gray} IDE & Лицензия & Windows & Linux & Разработка GUI\\\hline\hline
    Eclipse & EPL  & Да \cellcolor{Red}   & Да \cellcolor{Red}   & Да \cellcolor{Red} \\\hline
    IntelliJ Idea &ASL  & Да \cellcolor{Red}   & Да \cellcolor{Red}  & Да \cellcolor{Red} \\\hline
    KDevelop & GPL  & Нет \cellcolor{green}   & Да \cellcolor{Red}   & Нет \cellcolor{green}  \\\hline
    NetBeans &CDDL, GPL, LGPL  & Да \cellcolor{Red}   & Да \cellcolor{Red}   & Да \cellcolor{Red} \\\hline
    BlueJ & GPL  & Да \cellcolor{Red}   & Да \cellcolor{Red}   & Нет \cellcolor{green} \\\hline

    \end{tabular}
\end{table}

\section{Vim как IDE}
Vim - это клон с текстового редактора vi Билла Джоя для Unix c некоторыми дополнениями. Автор Vim, 
Брэм Мооленаар, основал его на исходном коде для переноса редактора Stevie на Amiga и выпустил
общедоступную версию в 1991 году. Vim разработан для использования как из интерфейса командной 
строки, так и как отдельное приложение с графическим пользовательским интерфейсом. Vim - это 
бесплатное программное обеспечение с открытым исходным кодом. Лицензия совместима с Стандартной
общественной лицензией GNU через специальный пункт, разрешающий распространение измененных копий 
«под GNU GPL версии 2 или любой более поздней версии».\\

С момента выпуска для Amiga кроссплатформенная разработка сделала Vim доступным для многих других 
систем. В 2006 году он был признан самым популярным редактором среди читателей Linux Journal; В 
2015 году опрос разработчиков Stack Overflow показал, что он стал третьим по популярности текстовым
редактором и пятой по популярности средой разработки в 2019 году.\\

Пусть Vim и является в первую очередь текстовым редактором, количество разработчиков, использующих 
его в качестве IDE, очень велико. Vim не загроможден ненужными функциями, в отличие от многих 
тяжеловесных IDE, большая часть функционала которых оказывается бесполезна для большинства 
пользователей. VIM позволяет настроить как общую конфигурацию для всех проектов, так и отдельную 
для каждого языка программирования. Есть возможность загрузить одни плагины для C, другие для Python 
и другие для Java.\\

Обычная IDE может имеет тесно интегрированную языковую поддержку и более практичные функции для рефакторинга кода. Однако, например, может оказаться, что программисту, пишущему код в основном на 
C++, возникла необходимость взглянуть на код Java. При наличии Vim не нужно устанавливать отдельную IDE и получить минимальную языковую поддержку без дополнительных действий. \\

В Vim все операции выполняются с клавиатуры, что помогает не отвлекаться от написания кода.\\ 

Vim - гибкий и практичный инструмент, но его нужно настроить заранее. IDE - в большинстве случаев
позволяет сразу приступить к работе. Vim является мощным и многофункциональным инструментом, в то 
время как IDE - мощный, но узкоспециализированный инструмент. Vim поддерживает сколько угодно языков, 
IDE - поддерживает 1-3 языка, но лучше, чем VIM.\\

Тем не менее Vim нужны плагины для выполнения некоторых функций, доступных в обычных IDE, которые 
не встроены в Vim изначально. Вот несколько плагинов Vim, которые делают его более похожим на IDE:\\

\subsection{Просмотр проекта / файлового дерева}
\begin{itemize}
    \item NERDTree - это плагин древовидного обозревателя для навигации по файловой системе;\\
    \item vtreeexplorer - это файловый менеджер на основе дерева;\\
    \item project дает "проектное" представление файлов, а не прямое представление файловой системы;\\
    \item ide отслеживает статус файлов (открытые / отредактированные / закрытые / только для чтения) 
        в проекте с помощью значков; автоматически строит и обновляет правила подсветки синтаксиса 
        на основе файлов проекта (C / C ++ / Java); избегает дублирования буфера.\\
    \item :help netrw нужен для получения информации о проводнике, поставляемом с Vim. По 
        умолчанию он не отображает файлы в виде дерева, но может использовать параметр 
        g: netrw\_liststyle. Он также предлагает полезные параметры сортировки файлов (по дате, 
        размеру, имени).\\
    \item Local\_vimrc управляет проектами как файлами в одном дереве каталогов;\\
    \item Projectionist обеспечивает детальную конфигурацию проекта с использованием «проекций»;\\
\end{itemize}

\subsection{Буфер / просмотр файлов}
\begin{itemize}
    \item bufexplorer позволяет перемещаться по открытым буферам;\\
    \item minibufexpl - элегантный проводник буферов; занимает очень мало места на экране;\\
    \item lookupfile - файлы поиска с использованием Vim7 ins-Completion;\\
    \item плагин Command-T, вдохновленный окном «Перейти к файлу», привязанным к Command-T в 
        TextMate;\\
    \item MRU - доступ к недавно открывавшимся файлам;\\
    \item ctrlp -  поиск с поддержкой регулярных выражений, предоставляет доступ ко всем функциям 
        с помощью ctrl-p;\\
\end{itemize}

\subsection{Просмотр кода}
\begin{itemize}
    \item taglist дает представление об исходном коде, который просматривается в данный момент;\\
    \item Tagbar похож на список тегов, но может упорядочивать теги по объему. Рекомендуется для 
        языков программирования с классами, например C ++, Java, Python;\\
    \item Indexer автоматически генерирует теги для всех файлов в проекте и поддерживает теги в 
        актуальном состоянии. Использует ctags. Хорошо работает с плагином project или 
        самостоятельно;\\
    \item CCTree - это обозреватель дерева вызовов, браузер исходного кода на основе Cscope и 
        анализатор потока кода;\\
    \item exUtility - глобальный поиск, поиск символов, отслеживание тегов;\\
    \item ShowMarks наглядно показывает расположение отметок;\\
    \item lh-tags автоматически обновляет базу данных ctags и предоставляет альтернативу 
        <<: tselect>> для навигации по коду;\\
\end{itemize}

\subsection{Написание кода}
\begin{itemize}
    \item AutoComplPop завершает код по мере ввода;\\
    \item YouCompleteMe - еще один плагин завершения;\\
    \item CRefVim Справочное руководство по C, специально разработанное для Vim;\\
\end{itemize}

\subsection{Функциональность Vim}
\begin{itemize}
    \item bufkill позволяет очистить буфер, не закрывая окно;\\
    \item undotree или gundo визуализирует дерево отмены;\\
    \item Surround упрощает удаление / изменение / добавление скобок / кавычек / XML-тегов / 
        многое другое;\\
\end{itemize}

\subsection{Компиляция}
\begin{itemize}
    \item vim-dispatch позволяет асинхронно запускать команды оболочки. При запуске компилятора 
        окно быстрого исправления будет заполнено любыми потенциальными ошибками;\\
    \item Build Tools Wrapper предоставляет способы компилировать программы (возможность 
        компилировать в фоновом режиме, на нескольких ядрах и т. д.), а также тестировать и 
        выполнять программы. Этот плагин также может фильтровать выходные данные компиляции на 
        лету. При компиляции проектов под CMake care плагин BTW позволяет переключать режим 
        компиляции (на самом деле каталог). Текущий режим компиляции (и имя проекта) будет 
        отображаться в строке состояния каждого буфера (в том числе буфера быстрого исправления) 
        через плагин airline;\\
\end{itemize}

\subsection{Отладка}
\begin{itemize}
    \item встроенный плагин отладчика терминала в комплекте;\\
    \item clewn реализует полную поддержку gdb в редакторе vim: точки останова, контрольные переменные, завершение команды gdb, окна сборки и т. д.;\\
    \item pyClewn похож на clewn, но написан на Python и также поддерживает pdb.;\\
    \item vim-debug, который создает интегрированную среду отладки в VIM;\\
    \item плагин gdbvim позволяет просматривать в vim, то, что отлаживается в gdb;\\
    \item vim-lldb: обеспечивает интеграцию отладки lldb;\\
\end{itemize}

\section{Emacs}
Emacs - это семейство текстовых редакторов, которые отличаются своей расширяемостью. В руководстве 
по наиболее широко используемому варианту, GNU Emacs, он описывается как «расширяемый, 
настраиваемый, самодокументирующийся редактор отображения в реальном времени». Разработка первого 
Emacs началась в середине 1970-х годов, и работа над его прямым потомком, GNU Emacs, активно 
продолжается до сих пор.\\

Emacs имеет более 10 000 встроенных команд, а его пользовательский интерфейс позволяет пользователю
объединять эти команды в макросы для автоматизации работы. Реализации Emacs обычно содержат диалект 
языка программирования Lisp, который обеспечивает возможность глубокого расширения, позволяя 
пользователям и разработчикам писать новые команды и приложения для редактора. Были написаны 
расширения для управления электронной почтой, файлами, схемами и RSS-потоками, а также клонами 
ELIZA, Pong, Conway's Life, Snake и Tetris.\\

Оригинальный EMACS был написан в 1976 году Дэвидом А. Муном и Гаем Л. Стилом-младшим как набор редакторов MACroS для редактора TECO. Он был вдохновлен идеями TECO-макроредакторов TECMAC и TMACS.\\

Самая популярная и наиболее портируемая версия Emacs - это GNU Emacs, созданная Ричардом Столлманом 
для проекта GNU. XEmacs - это вариант, ответвившийся на GNU Emacs в 1991 году. GNU Emacs и XEmacs
используют похожие диалекты Lisp и по большей части совместимы друг с другом. Разработка XEmacs 
неактивна.\\

Emacs, наряду с vi, является одним из двух основных соперников в традиционных войнах редакторов 
в культур Unix. Emacs - один из старейших бесплатных проектов с открытым исходным кодом, который 
все еще находится в стадии разработки.\\

\section{Сравнение Emacs и Vim}

\begin{table}[h!]
    \caption{Сравнение Emacs и Vim}
    \centering
    \begin{tabular}[h!]{| p{0.13\linewidth} | p{0.41\linewidth} | p{0.41\linewidth} |}
        \hline
        \rowcolor{gray} Функция & Vim & Emacs \\\hline\hline
        Нажатие клавиши & Vim сохраняет каждую комбинацию нажатых клавиш. Это создает 
        путь в дереве решений, который однозначно идентифицирует любую команду. & Команды Emacs - 
        это комбинации клавиш, для которых клавиши-модификаторы удерживаются, в то время как другие
        клавиши нажимаются; команда выполняется после полного ввода. Это по-прежнему формирует 
        дерево решений из команд, но не из отдельных нажатий клавиш. Пакет Emacs на основе vim 
        (дерево отмены) предоставляет пользовательский интерфейс для дерева.\\\hline

        Пользова-\linebreak тельская среда & Vim, как и Emacs, изначально использовался исключительно внутри 
        текстовой консоли, не предлагая графического интерфейса (GUI). Многие современные 
        производные vi, например MacVim и gVim включают графические интерфейсы. Однако поддержка
        пропорциональных шрифтов по-прежнему отсутствует. Также отсутствует поддержка шрифтов 
        разного размера в одном документе. & Emacs, изначально предназначенный для использования из
        консоли, имел поддержку графического интерфейса X11, добавленную в Emacs 18 и сделанную по
        умолчанию в версии 19. Текущие графические интерфейсы Emacs включают полную поддержку
        пропорционального интервала и изменения размера шрифта. Emacs также поддерживает встроенные
        изображения и гипертекст.\\\hline

        Поддержка языков и скриптов & Vim имеет элементарную поддержку других языков, кроме 
        английского. Современный Vim поддерживает Unicode, если используется с терминалом, 
        поддерживающим Unicode. & Emacs полностью поддерживает все Unicode-совместимые системы 
        письма и позволяет свободно смешивать несколько скриптов.\\\hline

        Интерфейс навигации & Vim использует различные режимы редактирования. В «режиме вставки» 
        клавиши вставляют символы в документ. В «нормальном режиме» (также известном как 
        «командный режим», не путать с «режимом командной строки», который позволяет пользователю 
        вводить команды), простые нажатия клавиш выполняют команды Vim. & Emacs использует 
        одновременное нажатие горячих клавиш. Клавиши или связки клавиш можно определить как 
        префиксные команды, которые переводят Emacs в режим ожидания дополнительных нажатий клавиш,
        составляющих окончание команды. Связки клавиш могут зависеть от режима, изменяя стиль
        взаимодействия. \\\hline
    \end{tabular}
\end{table}
