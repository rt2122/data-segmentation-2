\documentclass{article}
    \usepackage{amssymb}
    \usepackage[utf8]{inputenc}
    \usepackage[russian]{babel}
    \usepackage[left=2cm,right=2cm,
        top=2cm,bottom=2cm,bindingoffset=0cm]{geometry}
    \usepackage{hyperref}
    \hypersetup{
        colorlinks=true,
        linkcolor=blue,
        filecolor=magenta,      
        urlcolor=cyan,
    }
  \usepackage{graphicx}
  \usepackage{booktabs}
  \graphicspath{{pictures/}}
  \DeclareGraphicsExtensions{.pdf,.png,.jpg}
\usepackage{subcaption}
%\captionsetup{compatibility=false}

\begin{document}
\begin{center}{\hugeОтчет по курсовой работе за неделю\\}\end{center}
Дата: 18.2.2021\\
Научные руководители: Герасимов С.В., Мещеряков А.В.\\
Студент: Немешаева Алиса\\
Курс: 4\\

\renewcommand{\labelitemi}{$\blacksquare$}
\renewcommand\labelitemii{$\square$}
\begin{enumerate}
    \item Создан каталог скоплений act\_found\_pz\_rot28 для обучения новой модели. Этот каталог 
        создан на основе оригинального каталога ACT. Из ACT были исключены скопления, что уже 
        содержатся в каталоге planck\_z (на его основе создавалась обучающая выборка для предыдущих 
        моделей). Из оставшихся скоплений в каталог act\_found\_pz\_rot28 (сокращённо act\_found) 
        были добавлены те, что модель pz\_rot28 может детектировать (то есть те скопления, что 
        находятся в каталоге, созданном на основе модели pz\_rot28).\\
    \item После этого была создана обучающая выборка - образцовые карты сегментации на основе 
        каталогов planck\_z и act\_found. На основе этих карт и координат патчей pz\_only\_patches
        (созданы на основе каталога planck\_z) была создана модель pz\_act\_found.\\
    \item Далее была создана ещё одна модель - act\_found2. Её отличие от act\_found заключается в 
        том, что в обучающую и валидационную выборки были добавлены координаты патчей, в которых 
        находятся скопления из каталогов planck\_z и act\_found.\\
    \item На основе скоплений, которые может детектировать последняя модель в данных всех 
        используемых каталогов (PSZ2, MCXC, ACT, RedMaPPer), был создан ещё один каталог - 
        all\_found. На его основе также обучена модель (патчи были выбраны как для предыдущей 
        модели).\\
    \item Результаты всех полученных моделей сравниваются в приведенной таблице.\\
\end{enumerate}

\begin{table}
\begin{tabular}{lrrrrr}
\toprule
{}                          name &  planck\_z &  planck\_no\_z &    mcxcwp &     actwp &    fp \\
\midrule
                         pz14 &  0.939394 &     0.821429 &  0.155340 &  0.130435 &   949 \\
                         pz40 &  0.931818 &     0.785714 &  0.145631 &  0.130435 &  1180 \\
                     pz\_rot19 &  0.962121 &     0.750000 &  0.165049 &  0.145221 &  1439 \\
                     pz\_rot28 &  0.962121 &     0.785714 &  0.174757 &  0.141544 &  1287 \\
                     pz\_rot33 &  0.954545 &     0.785714 &  0.155340 &  0.154412 &  1379 \\
                     pz\_act10 &  0.901515 &     0.821429 &  0.097087 &  0.089202 &   624 \\
                     pz\_act14 &  0.825758 &     0.750000 &  0.058252 &  0.043478 &   557 \\
       pz\_act\_rot\_drop0.1\_ep9 &  0.931818 &     0.750000 &  0.155340 &  0.094518 &   605 \\
   pz\_act\_jan\_rot\_drop0.1\_ep6 &  0.954545 &     0.785714 &  0.106796 &  0.093750 &   693 \\
   pz\_act\_feb\_rot\_drop0.1\_ep5 &  0.969697 &     0.821429 &  0.116505 &  0.163603 &  1843 \\
  pz\_act\_feb\_rot\_drop0.2\_ep10 &  0.969697 &     0.821429 &  0.155340 &  0.136029 &  1424 \\
  pz\_act\_feb\_rot\_drop0.3\_ep14 &  0.946970 &     0.714286 &  0.145631 &  0.152574 &  1695 \\
          pz\_act\_q\_0.1\_0.9\_12 &  0.962121 &     0.821429 &  0.145631 &  0.139706 &  1039 \\
          pz\_act\_q\_0.1\_0.9\_14 &  0.962121 &     0.750000 &  0.165049 &  0.148897 &  1290 \\
               pz\_act\_found13 &  0.954545 &     0.785714 &  0.145631 &  0.139706 &  1232 \\
               pz\_act\_found16 &  0.946970 &     0.821429 &  0.126214 &  0.125000 &  1073 \\
               pz\_act\_found21 &  0.946970 &     0.750000 &  0.165049 &  0.136029 &  1307 \\
             pz\_all\_found2\_29 &  0.946970 &     0.821429 &  0.194175 &  0.158088 &  1341 \\
             pz\_all\_found2\_34 &  0.946970 &     0.821429 &  0.203883 &  0.161765 &  1564 \\
\bottomrule
\end{tabular}
\caption{Сравнение результатов последних моделей pz\_act\_found, pz\_act\_found2, pz\_all\_found 
    с предыдущими на валидационной области.}
\end{table}

\begin{table}
\begin{tabular}{lrrrrrr}
\toprule
{} &      PSZ2 &      MCXC &        RM &       ACT &     fp &    all \\
\midrule
pz\_rot\_28        &  0.916515 &  0.425129 &  0.056336 &  0.218832 &  21084 &  23331 \\
pz\_act\_found2\_22 &  0.929825 &  0.432014 &  0.056681 &  0.230751 &  19628 &  21946 \\
pz\_all\_found34   &  0.922565 &  0.438325 &  0.061009 &  0.235757 &  21018 &  23352 \\
\bottomrule
\end{tabular}
\caption{Сравнение наилучших каталогов на всём небе.}
\end{table}

Отчет согласован с научным руководителем.\\
Общее количество строк кода за эту неделю: 109\\
\href{https://github.com/rt2122/data-segmentation-2}{Репозиторий}\\ 
\end{document}
