\documentclass{article}
    \usepackage{amssymb}
    \usepackage[utf8]{inputenc}
    \usepackage[russian]{babel}
    \usepackage[left=2cm,right=2cm,
        top=2cm,bottom=2cm,bindingoffset=0cm]{geometry}
    \usepackage{hyperref}
    \hypersetup{
        colorlinks=true,
        linkcolor=blue,
        filecolor=magenta,      
        urlcolor=cyan,
    }

\begin{document}
\begin{center}{\hugeОтчет по курсовой работе за неделю\\}\end{center}
Дата: 25.9.2020\\
Научные руководители: Герасимов С.В., Мещеряков А.В.\\
Студент: Немешаева Алиса\\
Курс: 4\\

\renewcommand{\labelitemi}{$\blacksquare$}
\renewcommand\labelitemii{$\square$}
За прошедшую неделю была проведена следующая работа:\\
\begin{enumerate}
    \item Обучено еще несколько моделей, их отличие от предыдущей заключется в увеличении расстояния 
        скоплений от центров патчей. Была достигнута полнота 1.0 на тестовой области, 
        \hyperlink{https://github.com/rt2122/data-segmentation-2/blob/master/Planck\_Unet/scan\_planck\_z\_f8\_d0.8.ipynb}{графики}.\\
    \item Статьи:
        \begin{itemize}
            \item \hyperlink{https://arxiv.org/pdf/1612.01105.pdf}{Pyramid Scene Parsing Network}.\\
            \item \hyperlink{https://arxiv.org/pdf/1707.03718.pdf}{LinkNet}.\\
            \item \hyperlink{https://arxiv.org/pdf/2009.01907.pdf}{W-Net}.\\
        \end{itemize}
\end{enumerate}

Общее количество строк кода: 1502\\
\end{document}
