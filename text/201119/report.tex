\documentclass{article}
    \usepackage{amssymb}
    \usepackage[utf8]{inputenc}
    \usepackage[russian]{babel}
    \usepackage[left=2cm,right=2cm,
        top=2cm,bottom=2cm,bindingoffset=0cm]{geometry}
    \usepackage{hyperref}
    \hypersetup{
        colorlinks=true,
        linkcolor=blue,
        filecolor=magenta,      
        urlcolor=cyan,
    }
  \usepackage{graphicx}
  \graphicspath{{pictures/}}
  \DeclareGraphicsExtensions{.pdf,.png,.jpg}
\usepackage{subcaption}
%\captionsetup{compatibility=false}

\begin{document}
\begin{center}{\hugeОтчет по курсовой работе за неделю\\}\end{center}
Дата: 19.11.2020\\
Научные руководители: Герасимов С.В., Мещеряков А.В.\\
Студент: Немешаева Алиса\\
Курс: 4\\

\renewcommand{\labelitemi}{$\blacksquare$}
\renewcommand\labelitemii{$\square$}
\begin{enumerate}
    \item На этой неделе началась работа по созданию классификатора детектированных объектов для 
        создания ранжированных каталогов.\\
    \item Классификатор нужен для того, чтобы можно было генерировать каталоги с разной степенью 
        полноты и надежности.\\
    \item Был написан код для генерации обучающей выборки для классификатора.\\
    \item Начато обучение модели SVM для классификации.\\
    \item Продолжение работы над статьей по каталогам: исправление введения и главы <<Обзор 
        существующих решений>> 
\end{enumerate}

Отчет согласован с научным руководителем.\\
Общее количество строк кода за эту неделю: 184\\
\hyperlink{https://github.com/rt2122/data-segmentation-2}{Репозиторий}\\ 
\end{document}
